\documentclass[../main.tex]{subfiles} % Due to use of package subfiles

%%%%%%%%%%%%%%%%%%%%%%%%%%%%%%%%%%%%%%%%%%%%%%%%%%%%%%%%%%%%%%%%%%%%%%%%%%%%%%%%

\begin{document}

\chapter{Quantum field theory on a lattice} \label{chap:LatticeQFT}

\ldots




\section{Discretization}

\ldots




\section{Lattice Schwinger model}

\ldots




% \section{Equivalent spin formalism of the lattice Schwinger model}
\subsection{Mapping to spin system}

\ldots



\subsubsection{Jordan-Wigner transformation}

The Jordan-Wigner transformation is a transformation mapping fermionic creation and annihilation operators onto spin operators of the spin-\half system for one-dimensional lattices, thus we will use this to transform the fermions on the matters sites to spin step operators. This section will take its starting point in Refs. \cite{jordan-wigner_1928, banksSusskindKogut_StrongCopling_1976, panyella_masterThesis_2019}.

Remembering the definition of the step spin operators \cite{sakurai_modernQM_2017}
\begin{align}
    \sigma^\pm &= \frac{\sigma_x \pm i \sigma_y}{2} \: ,
\end{align}
one may easily find the anticommutator between these for the same lattice site to be $\anticommutator{\sigma_n^+}{\sigma_n^-} = 1$, as would also be expected from the creation and annihilation operators, $\anticommutator{\psi_n\dagger}{\psi_n} = 1$. Thus one might be tempted to use the direct mapping $\psi_n\dagger = \sigma_n^+$ and $\psi_n = \sigma_n^-$, but then the operators would commute for different lattice sites ($n \ne m$) \cite{susskind_latticeFermions_1977}, $\commutator{\psi_n\dagger}{\psi_m} = \commutator{\sigma_n^+}{\sigma_m^-} = 0$, which is not the case for fermionic operators, for which these must always anticommute \cite{sakurai_modernQM_2017}, $\anticommutator{\psi_n\dagger}{\psi_m} = 0$.

To accommodate the above stated problem we introduce the Jordan-Wigner transformation
\begin{align} \label{eq:Jordan-WignerTransformation}
\begin{split}
    \psi_n &= \prod_{m=0}^{n-1} (i\sigma_m^z) \sigma_n^- \: , \\
    \psi_n\dagger &= \prod_{m=0}^{n-1} (-i\sigma_m^z) \sigma_n^+ \: .
\end{split}
\end{align}
for which the added product part corresponds to all the lattice sites from one end of the lattice to the site concerned, thus ensuring the anticommutator, that the previous stated (direct) transformation did not satisfy. It can fairly easy be shown, that the Jordan-Wigner transformation ensures the correct anticommutator
\begin{align}
    \anticommutator{\psi_n}{\psi_m} &= (-1)^n i \anticommutator{\sigma_n^-}{\sigma_n^z} \prod_{l=n+1}^{m-1} (i \sigma_l^z) \sigma_m^- = 0 \: ,
\end{align}
by splitting the product parts\footnote{The product parts shall be split into into three parts: $[0,\ldots,n-1]$, $n$ and $[n+1,\ldots,m-1]$} and using $\anticommutator{\sigma_n^\pm}{\sigma_n^z} = 0$ and $\commutator{\sigma_n^\pm}{\sigma_m^z}$ (the operators only operate on their designated lattice site, thus operators with $n \ne m$ won't interfere with each other), while the transformation still preserves the anticommutator $\anticommutator{\psi_n\dagger}{\psi_n} = \anticommutator{\sigma_n^+}{\sigma_n^-} = 1$, due to the added product part cancelling with itself.



\subsubsection{Quantum Link Models}

\ldots




\end{document}