\documentclass[../main.tex]{subfiles} % Due to use of package subfiles

%%%%%%%%%%%%%%%%%%%%%%%%%%%%%%%%%%%%%%%%%%%%%%%%%%%%%%%%%%%%%%%%%%%%%%%%%%%%%%%%

\begin{document}

\chapter{Quantum field theory and gauge theory} \label{chap:ContinuumQFT}

\ldots



\section{Covariant tensor notation}

\ldots



\section{Lagrangian formalism of fields}

In quantum field theory the dynamics of a system is described using the Lagrangian formalism, but instead of considering the Lagrangian itself one consider the \emph{Lagrangian density} \index{Lagrangian density} $\L$, which is a function of one or more fields \emph{fields} \index{field} $\{\psi_i\}$ and their space- and time-derivatives $\{\partial_\mu \psi_i\}$.



\section{The continuum Schwinger model}

\subsection{Dirac field}

Due to our interest in examining confinement of quarks, which are spin-\half fermions, we will take as the starting point the most prominent fermionic field, namely the Dirac field. For this field the Lagrangian density \index{Lagrangian density!of Dirac field} is
\begin{align}
    \L_D &= \psibar (i \gamma^\mu \partial_\mu - m) \psi \: ,
\end{align}


\subsection{Local U(1) gauge invariance}

When considering systems one may often also consider their symmetries. Generally a symmetry \index{symmetry} is written as an invariance of the action, and thus of the Lagrangian density, with respect to transformation represented by operators \cite{peskin_introToQFT_1995}.




\end{document}