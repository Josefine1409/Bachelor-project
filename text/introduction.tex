\documentclass[../main.tex]{subfiles} % Due to use of package subfiles

%%%%%%%%%%%%%%%%%%%%%%%%%%%%%%%%%%%%%%%%%%%%%%%%%%%%%%%%%%%%%%%%%%%%%%%%%%%%%%%%

\begin{document}

\chapter{Introduction} \label{chap:Introduction}

% About QFT
Quantum field theory is a theoretical framework extending quantum mechanics by combining it with special relativity and classical field theory. While quantum mechanics is concerned with describing one or a few particles and relativistic quantum mechanics extends this focus to include the spin of the particles, quantum field theory is a theory enabling us to describe systems of many particles and also allows the treatment of fields, thus both particles and fields can be treated in the same framework. \cite{stanford_QFT}

% History of QFT (development of QED and QCD)
The work on developing the quantum field theory framework came shortly after the development of quantum mechanics, due to physicists wanting to not only describe particles but also fields in quantum mechanics. The foundation of the first quantum field theory, the quantum electrodynamics, is often said to be laid by Dirac for his work on quantisation of the electromagnetic field (combining quantum mechanics and field theory) and the relativistic theory of the the electron, the Dirac equation, (combining quantum mechanics and special relativity) \cite{stanford_historyOfQFT}. As quantum electrodynamics in the 1950s became a reliable and non-preliminary theory physicists began to create theories in analogy to this, for example quantum chromodynamics, and in 1973 the concept of colour charge as the source of the strong nuclear field was developed \cite{britannica_QCD}. There exists a considerably amount of quantum field theories, but the above mentioned theories, quantum electrodynamics and quantum chromodynamics, are two of the most known quantum field theories.

% Standard model -> Confinement and QCD
Today one of the most successful theories in physics is the so called \emph{Standard Model}\index{Standard Model} of elementary particle physics \cite{peskin_introToQFT_1995} describing three of the four fundamental forces: The strong nuclear force, the weak nuclear force and the electromagnetic force, but not the gravitational force. As it turns out the Standard Model is a gauge theory -- which is a type of quantum field theory -- resulting from the demand of local U(1)$\otimes$SU(2)$\otimes$SU(3) gauge invariance (see \cref{sec:ContinuumQFT_LocalU(1)GaugeInvariance}) \cite{peskin_introToQFT_1995, stanford_historyOfQFT}.
One of the beautiful key mechanisms of the Standard Model is the confinement of colour charge, also known as \emph{colour confinement}\index{colour confinement} or just confinement, which states that colourful particles as quarks cannot exist freely but only exist in hadrons \cite{peskin_introToQFT_1995, wilson_confinement_1974, griffiths_introToElementaryParticles_2008}. This is a consequence of the Standard Model incorporating quantum chromodynamics (local SU(3) gauge invariance). Wanting to examine the properties of quantum chromodynamics exactly is met with complications, since this quantum field theory is not analytically solvable \cite{fox_parallelComputingWorks_1994}.

% Schwinger model (as toy model)
Considering quantum electrodynamics and reducing it from the usual (3+1)-dimensions (three spatial and one temporal dimension) to (1+1)-dimensions one obtains the \emph{Schwinger model}\index{Schwinger model}. The (1+1)-dimensional Schwinger model is interesting for multiple reasons: Firstly it is analytically solvable, though merely for very light and very heavy fermions \cite{crewther_eigenvalueForSchwingerModel_1980}, and secondly it exhibits several of the desired properties of (3+1)-dimensional quantum chromodynamics including confinement \cite{hamer_massiveSchwingerModelOnLattice_1982}. Thus the (1+1)-dimensional Schwinger model yields an ideal candidate for a ''toy model'' of quantum chromodynamics.

% My project
My project have been concerning itself with the understanding of the theoretical framework of quantum field theory, and thus gauge theories, both in the continuum and on a lattice, such that I would be able to derive the lattice Schwinger model myself. In \cref{chap:ContinuumQFT} I take as a starting point Dirac's fermionic Lagrangian density and impose local U(1) gauge invariance leading to the derivation of quantum electrodynamics, and thus the continuum Schwinger model. In \cref{chap:LatticeQFT} the continuum Schwinger model is discretized using lattice gauge theory and then approximated by spin matrices, such that it can be modelled on a computer by a well known system. % Lastly \cref{chap:Confinement} serves as an outlook

% Natural units
Lastly it shall be mentioned that throughout this project we will be using the natural units $\hbar = c = 1$, thus length and time has the same unit and this being the reciprocal unit of energy and mass, $[\mathrm{length}] = [\mathrm{time}] = [\mathrm{energy}]^{-1} = [\mathrm{mass}]^{-1}$. This choise is due to it being convention when working with field theory, since we then are able to express most dimensionful quantities using only a single unit, and it simplifies our equations that we do not have to write the constants in every equation.


\end{document}