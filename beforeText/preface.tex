\documentclass[../main.tex]{subfiles} % Due to use of package subfiles

%%%%%%%%%%%%%%%%%%%%%%%%%%%%%%%%%%%%%%%%%%%%%%%%%%%%%%%%%%%%%%%%%%%%%%%%%%%%%%%%

\begin{document}

\chapter*{Preface}
\addcontentsline{toc}{chapter}{Preface}

Preface text


Before moving on, I would like to acknowledge my supervisor \Supervisor, \ldots

I would also like to thank Simon Panyella Pedersen for his very helpful email correspondence regarding the understanding of the increase of degrees of freedom in the gauge fields of the Schwinger model.

Finally, I would like to thank the people who took the time to proof-read and comment on this manuscript: \ldots. \ldots


% Looking at the distribution of pages in this document it would seem that I have spent more time reading the work of others rather than doing research myself. However, it is only by applying myself to the theories that I have come to understand them on a level where I can write long chapters detailing their content and operation, beyond what was actually necessary to perform the analyses and simulations that I report on in this thesis. Indeed, I regard the greatest achievement of my project to be the acquisition of a substantial amount of knowledge and understanding of superconducting circuits and quantum field theory. With this in mind I intended this thesis, in particular the theoretical chapters, as a notebook and manual to myself and hopefully others.

% Before moving on to the thesis proper I would like to acknowledge the competent and engaged supervision of Nikolaj T. Zinner. He has always been fast and committed to helping when prompted. His deep interest in applicable results of superconducting circuits and broad understanding of quantum field theory, has taught me to actually really care about what is realistic, and helped me navigate the overwhelming ocean of academic references. Likewise, I would like to acknowledge the helpful discussions at the weekly group meetings with Kasper Sangild Christensen, Niels Jakob Søe Loft, Stig Elkjær Rasmussen, Lasse Bjørn Kristensen, and Thomas Bækkegaard, and their readiness to answer questions in general. I would also like to thank Torsten Zache for his very helpful email correspondence regarding the analytical details of his article and my work with recreating them. Finally, I would like to thank the people who took the time to proof-read this manuscript.

% On a more personal level I would like to thank my delightful friends for their company in our years at Aarhus University. In particular in this final year, where everybody has been spending most of their time researching on their own, transitioning from drunken students to less-drunken office-dwellers, I find it important to have people to turn to for company, lest you go mad with boredom.


\end{document}