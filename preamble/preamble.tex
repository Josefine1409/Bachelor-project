%%%%%%%%%%%%%%%%%%%%%%%%%%%%%%%%%%%%%%%%%%%%%%%%%%%%%%%%%%%%%%%%%%%%%%%%%%%%%%%%
%%% DEFINITIONS
%%%%%%%%%%%%%%%%%%%%%%%%%%%%%%%%%%%%%%%%%%%%%%%%%%%%%%%%%%%%%%%%%%%%%%%%%%%%%%%%

% To use if-then-else statements
\usepackage{ifthen}
\newboolean{book} % Definition of the boolean ''book''
% Sets the boolean ''book'' to true if class is ''scrbook'' and false if other class
\makeatletter
    \@ifclassloaded{scrbook}{\setboolean{book}{true}}{\setboolean{book}{false}}
\makeatother

% Document properties
\title{\Title} % Title
\author{\Author} % Author
\date{\Month~\Year} % Hand in date


%%%%%%%%%%%%%%%%%%%%%%%%%%%%%%%%%%%%%%%%%%%%%%%%%%%%%%%%%%%%%%%%%%%%%%%%%%%%%%%%
%%% PACKAGES
%%%%%%%%%%%%%%%%%%%%%%%%%%%%%%%%%%%%%%%%%%%%%%%%%%%%%%%%%%%%%%%%%%%%%%%%%%%%%%%%

\usepackage[utf8]{inputenc} % Essential to keep!
\usepackage[british]{babel} % Language
\usepackage[
    bookmarks=true, % PDF-bookmarks
    bookmarksnumbered=true, % Numbered PDF-bookmarks
    bookmarksopen=true, % Bookmarks are opened
    bookmarksopenlevel=1, % Bookmarks opened but only the chapters
    bookmarksdepth=2 % Bookmarks include chapters, sections and subsection
]{hyperref} % Hyperlinks
\usepackage{physics} % Physics
\usepackage{nicefrac} % Nice in text fractions
\usepackage{lipsum} % Dummy text


%%%%%%%%%%%%%%%%%%%%%%%%%%%%%%%%%%%%%%%%%%%%%%%%%%%%%%%%%%%%%%%%%%%%%%%%%%%%%%%%
%%% FONTS
%%%%%%%%%%%%%%%%%%%%%%%%%%%%%%%%%%%%%%%%%%%%%%%%%%%%%%%%%%%%%%%%%%%%%%%%%%%%%%%%

\usepackage[T1]{fontenc} % Essential to keep!

% Linux Libertine
\usepackage{libertine}
\usepackage{libertinust1math}

\usepackage{dsfont} % For double lined letters and numbers (like has a mathbb-like commando working for numbers)

% URL's in the same font and size as the text around it
\urlstyle{same}


%%%%%%%%%%%%%%%%%%%%%%%%%%%%%%%%%%%%%%%%%%%%%%%%%%%%%%%%%%%%%%%%%%%%%%%%%%%%%%%%
%%% MARGIN
%%%%%%%%%%%%%%%%%%%%%%%%%%%%%%%%%%%%%%%%%%%%%%%%%%%%%%%%%%%%%%%%%%%%%%%%%%%%%%%%

\ifthenelse{\boolean{book}}{
    % Default book margins
    \usepackage[lmargin=142pt,rmargin=95pt,tmargin=127pt,bmargin=123pt]{geometry}
}{
    % Margins for only 2400 characters including spaces (from lipsum) on one page - article style
    \usepackage[lmargin=5cm,rmargin=5cm,tmargin=4.75cm,bmargin=3.8cm]{geometry}
}


%%%%%%%%%%%%%%%%%%%%%%%%%%%%%%%%%%%%%%%%%%%%%%%%%%%%%%%%%%%%%%%%%%%%%%%%%%%%%%%%
%%% DOCUMENT DEFINITIONS
%%%%%%%%%%%%%%%%%%%%%%%%%%%%%%%%%%%%%%%%%%%%%%%%%%%%%%%%%%%%%%%%%%%%%%%%%%%%%%%%

% Here the definitions of variables are defined
%%%%%%%%%%%%%%%%%%%%%%%%%%%%%%%%%%%%%%%%%%%%%%%%%%%%%%%%%%%%%%%%%%%%%%%%%%%%%%%%
%%% VARIABLE DEFINITIONS
%%%%%%%%%%%%%%%%%%%%%%%%%%%%%%%%%%%%%%%%%%%%%%%%%%%%%%%%%%%%%%%%%%%%%%%%%%%%%%%%

% The year and month the degree is awarded
\newcommand{\Year}[1]{\gdef\Year{#1}}
\newcommand{\Month}[1]{\gdef\Month{#1}}

\newcommand{\Date}[1]{\gdef\Date{#1}} % Date of hand in

% The full name of the degree
\newcommand{\Degree}[1]{\gdef\Degree{#1}}

% The short name of the degree (i.e. "bachelor" instead for the full name "bachelor of science")
\newcommand{\DegreeShort}[1]{\gdef\DegreeShort{#1}}

% The type of document (thesis/dissertation/project)
\newcommand{\Doctype}[1]{\gdef\Doctype{#1}}

% The title
\newcommand{\Title}[1]{\gdef\Title{#1}}

% The author's name
\newcommand{\Author}[1]{\gdef\Author{#1}}

% The author's e-mail (with clickable mail, such that it opens ones mail program; this is due to the href)
\newcommand{\AuthorEmail}[1]{\gdef\AuthorEmail{\href{mailto:#1}{#1}}}

% The univeristy name
\newcommand{\University}[1]{\gdef\University{#1}}

% The faculty name
\newcommand{\Faculty}[1]{\gdef\Faculty{#1}}

% The research group name
\newcommand{\Researchgroup}[1]{\gdef\Researchgroup{#1}}

% The department name
\newcommand{\Department}[1]{\gdef\Department{#1}}

% The address of the department
\newcommand{\DepartmentAddress}[1]{\gdef\DepartmentAddress{#1}}

% The name of the supervisor
\newcommand{\Supervisor}[1]{\gdef\Supervisor{#1}}

% The title of the supervisor
\newcommand{\SupervisorTitle}[1]{\gdef\SupervisorTitle{#1}}

% Subject one is studying
\newcommand{\SubjectOfStudy}[1]{\gdef\SubjectOfStudy{#1}} 

% Default values for the above variable definitions (only to try out the layout, so cannot be imported alongside documentPoperties.tex - only one of them at the time)
% %%%%%%%%%%%%%%%%%%%%%%%%%%%%%%%%%%%%%%%%%%%%%%%%%%%%%%%%%%%%%%%%%%%%%%%%%%%%%%%%
%%% DEFAULT VALUES OF VARIABLES
%%%%%%%%%%%%%%%%%%%%%%%%%%%%%%%%%%%%%%%%%%%%%%%%%%%%%%%%%%%%%%%%%%%%%%%%%%%%%%%%

\Year{1905}
\Month{January}
\Author{Author}
\AuthorEmail{email@domain.com}
\University{University}
\Faculty{Faclty}
\Department{Department}
\DepartmentAddress{
    Street name and number\\
    Postal code and city\\
    Country
}
\Researchgroup{Researchgroup}
\Title{This is the Title of the Document and it can be Multiple Lines if that is Wanted}
\Degree{Bachelor of Science}
\DegreeShort{Bachelor}
\Doctype{senior thesis}
\Supervisor{Supervisor}
\SupervisorTitle{Ph.D.}
\SubjectOfStudy{Subject one is studying}

% Actual values for the document
%%%%%%%%%%%%%%%%%%%%%%%%%%%%%%%%%%%%%%%%%%%%%%%%%%%%%%%%%%%%%%%%%%%%%%%%%%%%%%%%
%%% PROPERTIES OF THE DOCUMENT
%%%%%%%%%%%%%%%%%%%%%%%%%%%%%%%%%%%%%%%%%%%%%%%%%%%%%%%%%%%%%%%%%%%%%%%%%%%%%%%%

% Title, author, hand-in date
\Title{Confinement and String Breaking for the (1+1)-Dimensional Lattice Schwinger Model}
% \Title{The (1+1)-Dimensional Lattice Schwinger Model, Confinement and String Breaking}
\Author{Josefine Bjørndal Robl}
\Year{2020}
\Month{June}
\Date{15}

% Email of the author
\AuthorEmail{josefine@robl.dk}

% Thesis name
\Doctype{Bachelor's project}

% Subject of the thesis
\SubjectOfStudy{Physics}

% Degree, department, faculty, university
\Degree{Bachelor of Science}
\DegreeShort{Bachelor}
\Department{Department of Physics and Astronomy}
\Faculty{Faculty of Natural Sciences}
\University{Aarhus University}

% Address of the department
\DepartmentAddress{
    Ny Munkegade 120\\
    DK-8000 Aarhus C\\
    Denmark
}

% Advisor, advisor's title, research group
\Supervisor{Nikolaj Thomas Zinner}
\SupervisorTitle{associate professor} % Shall be written as it would in the middle of a text % The actual values


%%%%%%%%%%%%%%%%%%%%%%%%%%%%%%%%%%%%%%%%%%%%%%%%%%%%%%%%%%%%%%%%%%%%%%%%%%%%%%%%
%%% HEADER AND FOOTER
%%%%%%%%%%%%%%%%%%%%%%%%%%%%%%%%%%%%%%%%%%%%%%%%%%%%%%%%%%%%%%%%%%%%%%%%%%%%%%%%

% Import package for custom header and footer
\usepackage{fancyhdr}
\fancyhf{}

% Define \rightorleftmark, i.e. choose \leftmark if rightmark is empty
\makeatletter
\newcommand{\rightorleftmark}{%
  \begingroup\protected@edef\x{\rightmark}%
  \ifx\x\@empty
    \endgroup\leftmark
  \else
    \endgroup\rightmark
  \fi}
\makeatother

% Header and footer
\ifthenelse{\boolean{book}}{
    % Even pages (if two sided, i.e. ''book'' option)
    \fancyhead[LE]{\textsf{\nouppercase{\leftmark}}}
    \fancyhead[RE]{\textsf{\thepage}}
    
    % Odd pages (if two sided, i.e. ''book'' option)
    \fancyhead[RO]{\textsf{\nouppercase{\rightmark}}}
    \fancyhead[LO]{\textsf{\thepage}}
}{
    \fancyhead[L]{\textsf{\nouppercase{\rightorleftmark}}}
    \fancyhead[R]{\textsf{\thepage}}
}

% No horisontal lines to devide main section from header and footer
\renewcommand{\headrulewidth}{0pt}
\renewcommand{\footrulewidth}{0pt}

% Chapter pages uses the plain page style
\fancypagestyle{plain}{
    \fancyhf{}
    \fancyfoot[C]{\textsf{\thepage}}
}

% For the appendix (if use of backmatter)
\makeatletter
\renewcommand{\sectionmark}[1]{\markright{\thesection~~#1}}
\renewcommand{\chaptermark}[1]{\markboth{\if@mainmatter\chaptername\ \thechapter~~\fi#1}{}}
\makeatother

% Sets the page style
\pagestyle{fancy}

% Change how the header looks for the chapter name and section name respectively
\renewcommand{\chaptermark}[1]{\markboth{\chaptername~\thechapter~~$\bullet$~~#1}{}}
\renewcommand{\sectionmark}[1]{\markright{\thesection~~#1}}


%%%%%%%%%%%%%%%%%%%%%%%%%%%%%%%%%%%%%%%%%%%%%%%%%%%%%%%%%%%%%%%%%%%%%%%%%%%%%%%%
%%% TABLE OF CONTENTS
%%%%%%%%%%%%%%%%%%%%%%%%%%%%%%%%%%%%%%%%%%%%%%%%%%%%%%%%%%%%%%%%%%%%%%%%%%%%%%%%

% Set font for the chapter entries to serif and bold (addtokomafont is used to add the serif part to the already defined bold entries. setkomafont would have overruled)
\addtokomafont{chapterentry}{\rmfamily}

% TOC in PDF-bookmarks as a chapter (0)
% \BeforeTOCHead[toc]{{\pdfbookmark[0]{\contentsname}{toc}}}


%%%%%%%%%%%%%%%%%%%%%%%%%%%%%%%%%%%%%%%%%%%%%%%%%%%%%%%%%%%%%%%%%%%%%%%%%%%%%%%%
%%% CHAPTER STYLE
%%%%%%%%%%%%%%%%%%%%%%%%%%%%%%%%%%%%%%%%%%%%%%%%%%%%%%%%%%%%%%%%%%%%%%%%%%%%%%%%

% Change the font size of the chapter name and number
\usepackage{anyfontsize}

% Font of the chapter title
\setkomafont{chapter}{\mdseries\Huge}

% Chapter number in serif
\addtokomafont{chapterprefix}{\rmfamily}

% Chapter title placement down from the top
% \renewcommand*{\chapterheadstartvskip}{\vspace*{5\baselineskip}}

% Vertical space between chapter title and text
\renewcommand*{\chapterheadendvskip}{\vspace*{3\baselineskip}}

% Keep all the percentage tags right after the commands since there will otherwise be spaces that ruins the layout
\renewcommand*{\chapterformat}{%
  {\fontsize{15}{30}\scshape\chapappifchapterprefix{}}% % Changes 'CHAPTER' apperance
  \fontsize{50}{30}\selectfont\rlap{~\thechapter\autodot}% % Changes numbers apperance and \rlap shifts the number to the right
  \par\vspace{-.25em}{\rule{\textwidth}{.5pt}}% % Horisontal line
  \vspace{-.2em}% % Correct distance from line to chapter title
}

% Placement of title and ect. to the right
\renewcommand*{\raggedchapter}{\raggedleft}


%%%%%%%%%%%%%%%%%%%%%%%%%%%%%%%%%%%%%%%%%%%%%%%%%%%%%%%%%%%%%%%%%%%%%%%%%%%%%%%%
%%% BIBLIOGRAPHY
%%%%%%%%%%%%%%%%%%%%%%%%%%%%%%%%%%%%%%%%%%%%%%%%%%%%%%%%%%%%%%%%%%%%%%%%%%%%%%%%

% Proper quoting defined by the babel language
\usepackage{csquotes}

\usepackage[
    backend=biber,
    style=numeric-comp, % Citing [1-3] not [1,2,3]
    sorting=none, % No sorting of bibliography entries (i.e. order of appearance in text)
    % dashed=false, % No dash instead of author if the same author on article n and n+1
    maxbibnames=4, % Increase/decrease to include more/fewer authors in the bibliography
    doi=true, % Show DOI number
    url=true, % Show URL
    isbn=true, % Show ISBN
    hyperref=true, % Citations in text works as links
    urldate=comp, % Date visited shown as <#day>th <month> <#year> instead of <#date>/<#month>/<#year>
]{biblatex}

% Space in bibliography (change to compress/expand bibliography)
\setlength\bibitemsep{1.3\itemsep}

% Load the file with bibliographic information
\addbibresource{afterText/bibliography.bib}


%%%%%%%%%%%%%%%%%%%%%%%%%%%%%%%%%%%%%%%%%%%%%%%%%%%%%%%%%%%%%%%%%%%%%%%%%%%%%%%%
%%% INDEX
%%%%%%%%%%%%%%%%%%%%%%%%%%%%%%%%%%%%%%%%%%%%%%%%%%%%%%%%%%%%%%%%%%%%%%%%%%%%%%%%

\usepackage{imakeidx}
\makeindex[
    % columns = 2, % Number of columns (2 is default)
    % title = Index,
    options = -s preamble/indexPreamble.ist, % Each section in the index will be accompanied by their first letter (capital) in boldface - runs the command in the .ist document
    intoc, % Index in TOC
] % Lets one write to the index
% Uses the command \index{<entry>} and if nested \index{<entry>!<nested entry>}


%%%%%%%%%%%%%%%%%%%%%%%%%%%%%%%%%%%%%%%%%%%%%%%%%%%%%%%%%%%%%%%%%%%%%%%%%%%%%%%%
%%% REFERENCES
%%%%%%%%%%%%%%%%%%%%%%%%%%%%%%%%%%%%%%%%%%%%%%%%%%%%%%%%%%%%%%%%%%%%%%%%%%%%%%%%

% References where you don't have to write eg. ''see equation \re{<label>}'' but ''see \cref{<label>}''
\usepackage[english]{cleveref}

% Define \crefname for small first letter and \Crefname for capital first letter - used \cref{} and \Cref{}. Command goes ''\crefname{<type>}{<singular use>}{<plural use>}''

% Equations
\crefname{equation}{Eq.}{Eqs.}

% Figures
\crefname{figure}{Fig.}{Figs.}

% Tables
\crefname{table}{Table}{Tables}

% Chapters
\crefname{chapter}{Chapter}{Chapters}

% Sections
\crefname{section}{Section}{Sections}

% Subsections
\crefname{subsection}{Section}{Sections}


%%%%%%%%%%%%%%%%%%%%%%%%%%%%%%%%%%%%%%%%%%%%%%%%%%%%%%%%%%%%%%%%%%%%%%%%%%%%%%%%
%%% TABLES AND FIGURES
%%%%%%%%%%%%%%%%%%%%%%%%%%%%%%%%%%%%%%%%%%%%%%%%%%%%%%%%%%%%%%%%%%%%%%%%%%%%%%%%

% Packages
\usepackage{graphicx} % Images
\usepackage{float} % Images
\usepackage{epstopdf} % Converting EPS files to PDF
\usepackage{tikz} % Drawing
\usetikzlibrary{positioning} % For relative positioning of nodes
\usepackage[format=plain]{caption} % For no indent of caption text to fit end of caption name

% Set caption font
\addtokomafont{caption}{\sffamily\small} % Sans serif and smaller text

% Set caption label font
\setkomafont{captionlabel}{\scshape} % Small caps and sans serif
% \setkomafont{captionlabel}{\usekomafont{caption}} % For caption labels (eg. ''Figure X.Y'') in the same font as the captions


%%%%%%%%%%%%%%%%%%%%%%%%%%%%%%%%%%%%%%%%%%%%%%%%%%%%%%%%%%%%%%%%%%%%%%%%%%%%%%%%
%%% OWN HYPHENATIONS
%%%%%%%%%%%%%%%%%%%%%%%%%%%%%%%%%%%%%%%%%%%%%%%%%%%%%%%%%%%%%%%%%%%%%%%%%%%%%%%%

%%%%%%%%%%%%%%%%%%%%%%%%%%%%%%%%%%%%%%%%%%%%%%%%%%%%%%%%%%%%%%%%%%%%%%%%%%%%%%%%
%%% OWN HYPENATIONS
%%%%%%%%%%%%%%%%%%%%%%%%%%%%%%%%%%%%%%%%%%%%%%%%%%%%%%%%%%%%%%%%%%%%%%%%%%%%%%%%

% Hyphenation of words
\hyphenation{
    dif-fer-ent
    trans-for-ma-tion
    an-ti-com-mu-ta-tor
    an-ti-com-mu-ta-tors
    com-mu-ta-tor
    com-mu-ta-tors
    op-er-a-tor
    op-er-a-tors
    de-ri-va-tive
    de-ri-va-tives
    la-grangian
    la-grange
    for-ma-lism
    for-ma-lisms
    co-or-di-nate
    co-or-di-nates
    Hamil-to-ni-an
    Hamil-to-ni-ans
    Hamil-ton
    quan-tum
    with-out
}


%%%%%%%%%%%%%%%%%%%%%%%%%%%%%%%%%%%%%%%%%%%%%%%%%%%%%%%%%%%%%%%%%%%%%%%%%%%%%%%%
%%% OWN PHYSICS AND MATHEMATICS DEFINITIONS
%%%%%%%%%%%%%%%%%%%%%%%%%%%%%%%%%%%%%%%%%%%%%%%%%%%%%%%%%%%%%%%%%%%%%%%%%%%%%%%%

\renewcommand{\L}{\mathcal{L}} % Lagrangian density
\renewcommand{\H}{\mathcal{H}} % Hamiltonian density
\newcommand{\psibar}{\overline{\psi}} % Dirac adjoint of field psi
\newcommand{\half}{$\nicefrac{1}{2}$\space} % Nice half frac in text
\newcommand{\halfWOspace}{$\nicefrac{1}{2}$} % Nice half frac in text without space at the end, such that it can be used in a sentence before a dot, comma or ect.
\newcommand{\id}{\mathds{1}} % For identity 1

% Renewing \exp to contain round parentheses
\let\oldexp\exp
\renewcommand{\exp}[1]{\oldexp\left({#1}\right)}

% Renewing \dagger to be a superscript
\let\olddagger\dagger
\renewcommand{\dagger}{^\olddagger}